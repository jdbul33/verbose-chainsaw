\documentclass[11pt,twocolumn]{article}
\usepackage{graphicx}
%Use includegraphics{}for pics
\graphicspath{ {./South_Bend_311/} }
\title{Analysis of 3-1-1 Customer Call Center for South Bend, IN}
\author{John D. Bulger\\
\\
Valparaiso University\\
}
\begin{document}
\maketitle

\begin{abstract}
The city of South Bend, Indiana operates a call center that serves as a primary point of contact for citizens.  The call center handles topics from nearly all of the city's departments.  Open data, maintained by the city, contains several years worth of information.  An analysis of this data was conducted using Python 3.6.  The data was 
analyzed for patterns by time of year, department, and topics with varying methods.  The cleaned, manipulated, and explored data was then developed into an 
interactive dashboard using the Bokeh library.  In doing so, an interactive HTML file can be distributed to the city that can then be utilized, modified, and possibly connected 
directly to the data source.
\end{abstract}

\section{Background}
The city of South Bend, located in northern Indiana, established a citizen-accessible call center in February 2013.  The call center handles calls from citizens regarding almost every aspect of city interaction, including waste pick-up and removal, water billing and disconnections, 
and code enforcement.  By serving as a central hub for communication, the call center is able to consolidate much of the data regarding citizen/consumer issues.  
Much of this data is available on South Bend's open data portal at https://data-southbend.opendata.arcgis.com.  The guiding questions for this analysis includes seeking patterns and statistics on call volume and duration by timeframe, department, and topics.

\section{Data}
The data to be analyzed was acquired in three CSV files.  Two of the files are publicly available, and are also posted on the github repository for this project.  The third file was received directly from the city of South Bend, and as a result is not posted publicly at this time.
\par
The first file, referred to as the daily data, contains a daily summary of the call center data from the years 2013-2015.  It was organized with each record as a data, with each date containing information such as the number of calls presented, average wait time, average number of calls in queue, and number of abandoned calls.
\par
The second file, referred to as the the case data, contains approximately 480,000 rows of individual call data from 2013 through 2015.  Calls were logged anonymously with data such as date, duration, topic, and department.  This data format went obsolete when the new call system was implemented.
\par
The third file, referred to as the topic data, is the new storage method of the current phone system in use by the city, currently containing data from 2016-2018.  It contains much of the same attributes as the case data, but it standardized to reflect the city's use of knowledge base articles.  This ensures topics and departments are standardized across the list, allowing for efficient and accurate analysis.  Since this is the most recent data and is reflective of the city's current systems, this data was heavily relied on in analysis on a department and topic basis.

\section{Loading, Cleaning, \& Preprocessing the Data}
This entire analysis was completed using the Anaconda 5.2 distribution of Python 3.6 (Windows version).  The data was loaded into 3 separate dataframes through the use of pandas libarary.  The data was then inspected for missing data and reasonableness.  Very little data was missing, with the exception of a small percentage of observations in the case data frame (the older data).  These rows were missing a significant proportion of their respective attributes, and thus were dropped completely from the data.  Other data contained small amounts of missing data, which were easily imputed.
\par
Upon reading the csv files, pandas initialized most of the attributes as objects.  Preprocessing and transforming was conducted immediately to transform the attributes into more useful types, such as datetime and numeric objects.  Further processing was done throughout the project when necessary in order to provide a useful format.

\section{Call Data by Month}
An analysis of call volume by month was conducted using the daily data, which provides aggregate totals by day for a 2 year period. The relevant data was manipulated by taking the mean of the daily call volume for each day in its respective month.  Upon completion, the data was then plotted in a bar chart format, so as to quickly identify any differences.  This chart can be viewed in Figure 1.

\begin{figure}[h]
  \includegraphics{BARPLOTHERE.png}
  \caption{Plot of average daily call volume by month}
\end{figure}
\\

\subsection{Statistical Testing}
As an exercise in thoroughness, these findings were subjected to a two-sided t-test for statistical significance.  The call volumes of July and March were chosen, as these two months lie at the extremes on the above bar chart.  First, the corresponding records for each month were pulled from the data and stored in two separate lists.  Then, the scipy.stats package was used to conduct the test.  For the purposes of this test, the null hypothesis stated that "the two months have identical expected values."  The test disproved the null hypothesis, with a p-value of nearly zero.  %change
This supports the observation that the average daily call volume during these two months is indeed statistically significant.
\par
The same methodology was then applied to July and April, two of the highest volume months.  In this case, the test was unable to disprove the null hypothesis, showing that the daily volumes in these two months is not statistically different.  In looking at the bar chart, this is again in line with what the casual observer may infer.
\begin{figure}[h]
  \includegraphics{SOME VISUAL OF P VALUES HERE.png}
  \caption{A representation of the p-values from the above tests}
\end{figure}
\\

\section{Calls by Department}
One of the primary asks by the city was to determine call trends and statistics on a departmental level.  This was conducted primarily using the topic data, as the data contained accurate and complete information regarding departments.  In addition to being the most recent, this data also utilizes the new department/topic layout that the city has transitioned to.  The data was preprocessed to clean up the department titles prior to plotting for final dashboard by using a string replacement function in Python.  This function can be seen below:
\\
  \begin{verbatim}
  Insert func here
  \end{verbatim}
\\
\par
Additionally, the time format in the data file was read in as a string.  Upon investigating, the data was found to have an inconsistent format.  Duration was in the form "2:34" for two minutes, thirty-four seconds.  However, for longer calls it would be in the form "1:15:56" for one hour, fifteen minutes, and fifty-six seconds.  While beneficial visually, it would not function for any analysis.  This data was converted into seconds, stored as a numeric data type, using the following function:
\\
  \begin{verbatim}
  Insert func here
  \end{verbatim}
\\
\subsection{Analysis}
Calls by department were then analyzed by average duration.  While this analysis yielded insightful results, which could surely be of use to the call center.  Furthermore, the total number of calls by department was then calculated.  As expected, the department order in each analysis differed.  As a measure of total call volume, the department totals were calculated as the sum of total call time for that department.  All three insights are presented to the city in the final dashboard, but perhaps the total volume visual could be the most useful, especially in matters concerning budgeting.
\begin{figure}[h]
  \includegraphics{CALLVOLPYPLOTHERe.png}
  \caption{A visual of the final pie chart of total volume by department}
 \end{figure}
\\
 
\section{Calls by Topic}
Analyzing calls by topic proved to be one of the most challenging and interesting aspects of this analysis.  The same data was used as for the departmental analysis, given the homogenity of topic format.  Among this dataset, 426 unique call topics existed.  The data was then processed to show total call count and average call duration by topics.  This data was then merged with a seperate dataframe, indexed by topic with department as the only attribute.  Once completed, the resulting dataframe contained topic, average call duration, and corresponding department.  At this point, the data was ready for analysis.
\subsection{Duration}
Call duration by topics was calculated in the dataframe creation.  However, when sorted so as to see the extremes, several topics with a low volume rose to the top of average duration, some with only one total call.  In a sense, such records can be viewed as outliers, and were thus removed from reporting in the final dashboard.  In the end, only topics with a total volume of more than 50\% of the mean were included in this analysis.  By reducing the topics analyzed, the call center would be able to focus on topics that take a notable duration to resolve, but that also occur frequently enough to warrant attention.  The code, however, can be easily adjusted to a different threshold, if management so desires.
\begin{figure}[h]
  \includegraphics{BARPLOT.png}
  \caption{Durations by topic}
\end{figure}
\\
\subsection{Volume}
The total number of calls by topic was then determined using the same dataframe as for duration analysis.  While the most frequent call topics were quickly identified, significant overlap with the highest volume departments.  As a result, the volume by topic in and of itself would not serve much use to the city.  This analysis evolved into a by-topic volume analysis within the top several departments identified in Figure 3.  In order to visualize this data in the most effective way, a jitterplot was incorporated into the final dashboard.
\begin{figure}[h]
  \includegraphics{jitterplot.png}
  \caption{Jitterplot of call volume by topic for the busiest departments.  Hovertool is utilized to show topic}
\end{figure}

\section{Creation of Final Dashboard}
After completion of the above analysis, following guiding questions set forth by the city, the final dashboard was constructed.  The Bokeh package was used to create interactive visualizations, which could conveniently be packaged into a HTML file and sent to the call center management.  In all, the dashboard includes visualizations of:
\begin{itemize}
  \item{Pie chart of total call time by department}
  \item{Histogram of call volume by day of week, overlaid, with legend providing hide/show capabilities}
  \item{Bar chart of topics with longest average call duration}
  \item{Bar chart of topics with shortest average call duration}
  \item{Jitterplot of call volume by topic within top departments}
  \item{Bar chart of call volume by month}
\end{itemize}

\section{Conclusions}

Goodbye!


\end{document}
