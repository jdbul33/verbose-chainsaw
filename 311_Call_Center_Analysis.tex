\documentclass[11pt,twocolumn]{article}
\usepackage{graphicx}
%Use includegraphics{}for pics
\graphicspath{ {./South_Bend_311/} }
\title{Analysis of 3-1-1 Customer Call Center for South Bend, IN}
\author{John D. Bulger\\
\\
Valparaiso University\\
}
\begin{document}
\maketitle

\begin{abstract}
The city of South Bend, Indiana operates a call center that serves as a primary point of contact for citizens.  The call center handles topics from nearly all of the city's departments.  Open data, maintained by the city, contains several years worth of information.  An analysis of this data was conducted using Python 3.6.  The data was 
analyzed for patterns by time of year, department, and topics with varying methods.  The cleaned, manipulated, and explored data was then developed into an 
interactive dashboard using the Bokeh library.  In doing so, an interactive HTML file can be distributed to the city that can then be utilized, modified, and possibly connected 
directly to the data source.
\end{abstract}

\section{Background}
The city of South Bend, located in northern Indiana, established a citizen-accessible call center in 2013.  %Make sure this is correct
The call center handles calls from citizens regarding almost every aspect of city interaction, including waste pick-up and removal, water billing and disconnections, 
and code enforcement.  By serving as a central hub for communication, the call center is able to consolidate much of the data regarding citizen/consumer issues.  
Much of this data is available on South Bend's open data portal at https://data-southbend.opendata.arcgis.com.  The guiding questions for this analysis includes seeking patterns and statistics on call volume and duration by timeframe, department, and topics.

\section{Data}
The data to be analyzed was acquired in three CSV files.  Two of the files are publicly available, and are also posted on the github repository for this project.  The third file was received directly from the city of South Bend, and as a result is not posted publicly at this time.
\par
The first file, referred to as the daily data, contains a daily summary of the call center data from the years 2013-2015.  It was organized with each record as a data, with each date containing information such as the number of calls presented, average wait time, average number of calls in queue, and number of abandoned calls.
\par
The second file, referred to as the the case data, contains approximately 480,000 rows of individual call data from 2013 through 2015.  Calls were logged anonymously with data such as date, duration, topic, and department.  This data format went obsolete when the new call system was implemented.
\par
The third file, referred to as the topic data, is the new storage method of the current phone system in use by the city, currently containing data from 2016-2018.  It contains much of the same attributes as the case data, but it standardized to reflect the city's use of knowledge base articles.  This ensures topics and departments are standardized across the list, allowing for efficient and accurate analysis.  Since this is the most recent data and is reflective of the city's current systems, this data was heavily relied on in analysis on a department and topic basis.

\section{Loading, Cleaning, \& Preprocessing the Data}
This entire analysis was completed using the Anaconda 5.2 distribution of Python 3.6 (Windows version).  The data was loaded into 3 separate dataframes through the use of pandas libarary.  The data was then inspected for missing data and reasonableness.  Very little data was missing, with the exception of a small percentage of observations in the case data frame (the older data).  These rows were missing a significant proportion of their respective attributes, and thus were dropped completely from the data.  Other data contained small amounts of missing data, which were easily imputed.
\par
Upon reading the csv files, pandas initialized most of the attributes as objects.  Preprocessing and transforming was conducted immediately to transform the attributes into more useful types, such as datetime and numeric objects.  Further processing was done throughout the project when necessary in order to provide a useful format.

\section{Call Data by Month}
Talk about monthly approach
\par
Talk about statistical testing and findings

\section{Calls by Department}
Brief Intro
\subsection{Duration of Calls by Department}
More stuff, talk about resolution
\subsection{Number of Calls by Department}
More info, these sections need basic figures from matplotlib input here

\section{Calls by Topic}
Intro
\subsection{Patterns by Topic}
What is busiest, where is time spent dealing with 


\section{Creation of Final Dashboard}

\section{Conclusions}

Goodbye!


\end{document}
